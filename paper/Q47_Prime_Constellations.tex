\documentclass[12pt,a4paper]{article}

% ============================================================
% PACKAGES
% ============================================================
\usepackage[utf8]{inputenc}
\usepackage[T1]{fontenc}
\usepackage{amsmath,amssymb,amsthm}
\usepackage{mathtools}
\usepackage{graphicx}
\usepackage{booktabs}
\usepackage{array}
\usepackage{multirow}
\usepackage{longtable}
\usepackage{hyperref}
\usepackage{xcolor}
\usepackage{geometry}
\usepackage{fancyhdr}
\usepackage{enumitem}
\usepackage{natbib}
\usepackage{caption}
\usepackage{subcaption}
\usepackage{float}

% Page geometry
\geometry{margin=1in}

% Graphics path
\graphicspath{{./}{../figures/}}

% Hyperlinks
\hypersetup{
    colorlinks=true,
    linkcolor=blue!70!black,
    citecolor=green!50!black,
    urlcolor=blue!70!black
}

% Theorem environments
\newtheorem{theorem}{Theorem}[section]
\newtheorem{lemma}[theorem]{Lemma}
\newtheorem{proposition}[theorem]{Proposition}
\newtheorem{corollary}[theorem]{Corollary}
\newtheorem{conjecture}[theorem]{Conjecture}
\theoremstyle{definition}
\newtheorem{definition}[theorem]{Definition}
\newtheorem{example}[theorem]{Example}
\theoremstyle{remark}
\newtheorem{remark}[theorem]{Remark}

% Custom commands
\newcommand{\Q}{\mathcal{Q}}
\newcommand{\N}{\mathbb{N}}
\newcommand{\Z}{\mathbb{Z}}
\newcommand{\R}{\mathbb{R}}
\newcommand{\C}{\mathbb{C}}
\DeclareMathOperator{\ord}{ord}

% ============================================================
% TITLE
% ============================================================
\title{
    \textbf{Polynomial Constellations in Deep Arithmetic Space:}\\[0.5em]
    \large Empirical Analysis of Prime $k$-Tuples and the Bateman-Horn Heuristic\\
    for $Q(n) = n^{47} - (n-1)^{47}$
}

\author{
    Ruqing Chen\\
    \small GUT Geoservice Inc., Montreal, Canada\\
    \small \texttt{ruqing@hotmail.com}
}

\date{February 2026}

% ============================================================
% DOCUMENT
% ============================================================
\begin{document}

\maketitle

% ============================================================
% ABSTRACT
% ============================================================
\begin{abstract}
We present a computational analysis of the prime-generating polynomial $Q(n) = n^{47} - (n-1)^{47}$ for $n \le 2 \times 10^9$, accumulating a dataset of $17{,}908{,}247$ strong probable primes (each passing 25 rounds of Miller-Rabin testing) with up to 430 decimal digits. Our study reveals structured clustering phenomena we term \emph{Polynomial Constellations}: we identify $170{,}346$ consecutive pairs, $1{,}691$ triples, and $14$ quadruplets---instances where four consecutive integers $(n, n+1, n+2, n+3)$ all generate probable primes exceeding $10^{400}$.

We prove a \emph{Structural Exclusion Theorem}: the polynomial constraints strictly forbid twin primes of the form $(p, p+2)$ due to an inherent modular lock ($Q(n) \equiv 1 \pmod{3}$), while permitting consecutive prime generation at an observed consecutive-pair density of approximately $0.95\%$.

The observed prime counts are consistent with Bateman-Horn predictions using a correction factor $C_Q = 8.7 \pm 0.1$. No claim is made that this numerical agreement constitutes evidence for the Bateman-Horn conjecture itself; rather, this work should be read as an empirical study of how such heuristics manifest within a restricted computational regime.

\medskip
\noindent\textbf{Keywords:} Bateman-Horn heuristic, polynomial primes, prime constellations, experimental number theory, computational mathematics

\medskip
\noindent\textbf{MSC 2020:} 11N32, 11N05, 11Y11, 11A41
\end{abstract}



% ============================================================
% 1. INTRODUCTION
% ============================================================
\section{Introduction}

\subsection{Background and Motivation}

The distribution of prime numbers generated by polynomials $f(n)$ remains one of the most challenging frontiers in analytic number theory. While Dirichlet's theorem completely describes the linear case $f(n) = an + b$ with $\gcd(a,b) = 1$, the behavior of higher-degree polynomials is governed only by heuristics, most notably the Bateman-Horn conjecture \cite{bateman1962}.

The landmark result of Zhang \cite{zhang2014} proving bounded gaps between consecutive primes, subsequently refined by Maynard \cite{maynard2015} and the Polymath project led by Tao \cite{polymath2014}, demonstrated that prime distribution possesses deep structural properties beyond pure randomness. A natural question emerges:

\begin{quote}
\emph{Do these structural clustering properties persist, or even intensify, when primes are constrained to the values of a sparse, high-degree polynomial?}
\end{quote}

\subsection{The Polynomial $Q(n)$}

In this paper, we investigate the degree-46 polynomial:
\begin{equation}
    Q(n) = n^{47} - (n-1)^{47} = \sum_{k=0}^{46} \binom{47}{k} (-1)^{46-k} n^k
    \label{eq:Q_definition}
\end{equation}

This polynomial exhibits several notable properties:
\begin{enumerate}
    \item \textbf{Rapid growth:} $Q(n) \sim 47 \cdot n^{46}$, generating 430-digit primes at $n \approx 2 \times 10^9$.
    \item \textbf{Algebraic structure:} $Q(n)$ is related to the derivative of $x^{47}$ and shares properties with cyclotomic polynomials.
    \item \textbf{Modular behavior:} We prove that $Q(n) \equiv 1 \pmod{3}$ for all $n \ge 2$, creating a strict ``arithmetic resonance.''
\end{enumerate}

\subsection{Why This Study Matters}

Polynomial constellations offer a controlled environment in which Bateman-Horn type heuristics---normally formulated as asymptotic statements---can be confronted with finite-scale data, including secondary statistics beyond mere counting functions. In this sense, the present work should be read not as evidence for the Bateman-Horn conjecture itself, but as an empirical study of how its predictions manifest within a restricted computational regime.

The specific choice of exponent 47 (a prime) creates nontrivial modular constraints that allow us to:
\begin{itemize}
    \item Prove exact exclusion results (Theorem \ref{thm:exclusion})
    \item Test whether clustering phenomena predicted by heuristics appear at the expected rates
    \item Observe the interplay between algebraic structure and prime density
\end{itemize}

\subsection{Main Contributions}

Our investigation, spanning $n \le 2 \times 10^9$ with $17{,}908{,}247$ strong probable primes, yields three principal findings:

\begin{enumerate}
    \item \textbf{Exclusion Theorem:} We prove that $(Q(n), Q(n)+2)$ can \emph{never} both be prime, establishing a fundamental obstruction to twin primes in this sequence (Theorem \ref{thm:exclusion}).
    
    \item \textbf{Polynomial Constellations:} Despite the exclusion of twin primes, we discover 14 ``quadruplets''---instances where $Q(n), Q(n+1), Q(n+2), Q(n+3)$ are \emph{all} prime (Section \ref{sec:quadruplets}).
    
    \item \textbf{Bateman-Horn Consistency:} We compute a correction factor $C_Q = 8.7 \pm 0.1$ and observe that our prime counts are consistent with theoretical predictions within numerical uncertainty (Section \ref{sec:bateman_horn}).
\end{enumerate}

\subsection{Organization}

Section \ref{sec:exclusion} proves the Structural Exclusion Theorem. Section \ref{sec:resonance} analyzes the arithmetic resonance and modular distribution. Section \ref{sec:quadruplets} presents the 14 prime quadruplets. Section \ref{sec:bateman_horn} computes the Bateman-Horn correction factor. Section \ref{sec:counterexample} describes our minimal counterexample search. Section \ref{sec:conclusion} discusses implications and limitations.


% ============================================================
% 2. THE STRUCTURAL EXCLUSION THEOREM
% ============================================================
\section{The Structural Exclusion Theorem}
\label{sec:exclusion}

The first major result demonstrates that the polynomial structure \emph{forbids} certain prime configurations.

\subsection{Main Result}

\begin{theorem}[Right Twin Exclusion Law]
\label{thm:exclusion}
For all integers $n \ge 2$, the pair $(Q(n), Q(n)+2)$ cannot both be prime.
\end{theorem}

\begin{proof}
We show that $Q(n) \equiv 1 \pmod{3}$ for all $n \ge 2$ by analyzing each residue class.

First, note that for any $a \not\equiv 0 \pmod{3}$, Fermat's Little Theorem gives $a^2 \equiv 1 \pmod{3}$. Since $47 = 2 \times 23 + 1$ is odd:
\[
a^{47} = (a^2)^{23} \cdot a \equiv 1 \cdot a = a \pmod{3}
\]

We now consider three cases based on $n \pmod{3}$:

\textbf{Case 1:} $n \equiv 0 \pmod{3}$.\\
Then $n^{47} \equiv 0$ and $(n-1)^{47} \equiv (-1)^{47} = -1 \pmod{3}$.\\
Thus $Q(n) \equiv 0 - (-1) = 1 \pmod{3}$.

\textbf{Case 2:} $n \equiv 1 \pmod{3}$.\\
Then $n^{47} \equiv 1$ and $(n-1)^{47} \equiv 0^{47} = 0 \pmod{3}$.\\
Thus $Q(n) \equiv 1 - 0 = 1 \pmod{3}$.

\textbf{Case 3:} $n \equiv 2 \pmod{3}$.\\
Then $n^{47} \equiv (-1)^{47} = -1$ and $(n-1)^{47} \equiv 1^{47} = 1 \pmod{3}$.\\
Thus $Q(n) \equiv -1 - 1 = -2 \equiv 1 \pmod{3}$.

In all cases, $Q(n) \equiv 1 \pmod{3}$. Consequently:
\[
Q(n) + 2 \equiv 1 + 2 = 3 \equiv 0 \pmod{3}
\]

Since $Q(n) > 3$ for all $n \ge 2$, we have $Q(n) + 2 > 5$, making $Q(n) + 2$ divisible by 3 and strictly greater than 3, hence composite. Therefore $(Q(n), Q(n)+2)$ cannot both be prime.
\end{proof}

\subsection{Empirical Verification}

We verified Theorem \ref{thm:exclusion} across our entire dataset:

\begin{center}
\begin{tabular}{lc}
\toprule
\textbf{Metric} & \textbf{Value} \\
\midrule
Total $Q(n)$ primes tested & $17{,}908{,}247$ \\
Right twins $(Q(n), Q(n)+2)$ found & \textbf{0} \\
Theoretical prediction & 0 \\
\bottomrule
\end{tabular}
\end{center}

\begin{remark}
This theorem demonstrates that algebraic structure can override probabilistic expectations. While Zhang-Maynard-Tao results establish the \emph{existence} of bounded gaps in the general primes, our result establishes a \emph{prohibition} within a structured subsequence. We use the term ``Structural Exclusion'' to denote a deterministic modular obstruction specific to this polynomial, without implying any global distributional constraint.
\end{remark}

Figure \ref{fig:twin_exclusion} visualizes this exclusion principle alongside the permitted configurations.

\begin{figure}[H]
    \centering
    \includegraphics[width=0.95\textwidth]{twin_prime_analysis.pdf}
    \caption{Visual evidence of the Structural Exclusion Theorem. \textbf{Top-left:} The complete absence of right twins $(Q(n), Q(n)+2)$ contrasted with the existence of left twins. \textbf{Top-right:} Gap distribution showing that $k=2$ (traditional twins) is blocked while $k=4, 6, 12, \ldots$ are permitted by the mod-3 constraint. \textbf{Bottom:} Summary statistics confirming the theorem across 17.9 million samples.}
    \label{fig:twin_exclusion}
\end{figure}


% ============================================================
% 3. ARITHMETIC RESONANCE
% ============================================================
\section{Arithmetic Resonance and Modular Distribution}
\label{sec:resonance}

\subsection{General Modular Behavior}

The modular congruence $Q(n) \equiv 1 \pmod{3}$ is a special case of a broader phenomenon.

\begin{proposition}
\label{prop:mod_general}
For any prime $\ell > 2$, let $e = 47 \mod (\ell - 1)$. Then:
\[
Q(n) \equiv n^e - (n-1)^e \pmod{\ell}
\]
\end{proposition}

\begin{proof}
By Fermat's Little Theorem, $a^{\ell-1} \equiv 1 \pmod{\ell}$ for $\gcd(a, \ell) = 1$. Writing $47 = q(\ell-1) + e$ for some quotient $q$:
\[
a^{47} = a^{q(\ell-1)+e} = (a^{\ell-1})^q \cdot a^e \equiv a^e \pmod{\ell}
\]
The result follows immediately. Note that this exponent reduction describes the modular image of $Q(n)$, but does not by itself imply equidistribution among non-zero residues.
\end{proof}

\subsection{Quadratic Residue Distribution}

We computed the quadratic residue distribution of $Q(n)$ primes modulo small primes $\ell$:

\begin{table}[H]
\centering
\caption{Quadratic Residue Distribution of $Q(n)$ Primes}
\label{tab:qr_distribution}
\begin{tabular}{ccccc}
\toprule
$\ell$ & $e = 47 \mod (\ell-1)$ & Theoretical QR\% & Observed QR\% & Deviation \\
\midrule
3 & 1 & 100.0\% & 100.0\% & 0.0\% \\
5 & 3 & 60.0\% & 57.6\% & $-2.4\%$ \\
7 & 5 & 71.4\% & 71.2\% & $-0.2\%$ \\
11 & 7 & 50.0\% & 45.5\% & $-4.5\%$ \\
13 & 11 & 50.0\% & 51.2\% & $+1.2\%$ \\
17 & 15 & 56.3\% & 53.2\% & $-3.1\%$ \\
19 & 11 & 61.1\% & 60.6\% & $-0.5\%$ \\
23 & 3 & 50.0\% & 49.8\% & $-0.2\%$ \\
\bottomrule
\end{tabular}
\end{table}

The perfect agreement at $\ell = 3$ confirms the algebraic lock; the agreement within numerical uncertainty at other primes validates our computational methodology.

\subsection{Visualization of Arithmetic Resonance}

Figure \ref{fig:resonance} illustrates the effect of arithmetic resonance on the distribution of $Q(n)$ primes.

\begin{figure}[H]
    \centering
    \includegraphics[width=0.95\textwidth]{arithmetic_resonance_visualization.pdf}
    \caption{Arithmetic Resonance in $Q(n)$ primes. \textbf{Left:} Theoretical vs.\ observed quadratic residue ratios for small primes $\ell$. The perfect match at $\ell=3$ (100\% QR ratio) demonstrates the algebraic lock. \textbf{Right:} $\chi^2$ test results showing nontrivial deviation from uniform distribution at all tested moduli.}
    \label{fig:resonance}
\end{figure}


% ============================================================
% 4. THE ORION QUADRUPLETS
% ============================================================
\section{Prime Quadruplets: The 14 Dense Clusters}
\label{sec:quadruplets}

\subsection{Definition and Significance}

\begin{definition}[Polynomial Prime $k$-Tuple]
A \emph{polynomial prime $k$-tuple} is a sequence of $k$ consecutive integers $(n, n+1, \ldots, n+k-1)$ such that $Q(n), Q(n+1), \ldots, Q(n+k-1)$ are all prime.
\end{definition}

Our computational search discovered the following counts:

\begin{table}[H]
\centering
\caption{Distribution of Polynomial Prime $k$-Tuples for $n \le 2 \times 10^9$}
\label{tab:k_tuples}
\begin{tabular}{ccc}
\toprule
$k$ & Count & Terminology \\
\midrule
2 & 170,346 & Consecutive Pairs \\
3 & 1,691 & Triples \\
4 & \textbf{14} & Quadruplets (Dense Clusters) \\
5 & 0 & --- \\
\bottomrule
\end{tabular}
\end{table}

\subsection{The 14 Prime Quadruplets}

These dense clusters are distributed throughout the search range, with primes ranging from 380 to 430 digits.

\begin{table}[H]
\centering
\caption{Complete List of Prime Quadruplets for $Q(n) = n^{47} - (n-1)^{47}$}
\label{tab:quadruplets}
\begin{tabular}{rlrr}
\toprule
\# & Starting $n$ & Approx.\ Digits of $Q(n)$ & $n / 10^9$ \\
\midrule
1 & 117,309,848 & 380 & 0.117 \\
2 & 136,584,738 & 385 & 0.137 \\
3 & 218,787,064 & 390 & 0.219 \\
4 & 411,784,485 & 400 & 0.412 \\
5 & 423,600,750 & 401 & 0.424 \\
6 & 523,331,634 & 405 & 0.523 \\
7 & 640,399,031 & 408 & 0.640 \\
8 & 987,980,498 & 415 & 0.988 \\
9 & 1,163,461,515 & 420 & 1.163 \\
10 & 1,370,439,187 & 423 & 1.370 \\
11 & 1,643,105,964 & 426 & 1.643 \\
12 & 1,691,581,855 & 427 & 1.692 \\
13 & 1,975,860,550 & 429 & 1.976 \\
14 & 1,996,430,175 & 430 & 1.996 \\
\bottomrule
\end{tabular}
\end{table}

Figure \ref{fig:quadruplets} visualizes the distribution of these events across the search range.

\begin{figure}[H]
    \centering
    \includegraphics[width=0.9\textwidth]{figure1_quadruplets.pdf}
    \caption{Distribution of the 14 prime quadruplets. Each star represents a quadruplet---four consecutive integers $n, n+1, n+2, n+3$ all generating probable primes of approximately 380--430 digits. The dashed line shows the theoretical growth $\approx 46 \log_{10}(n)$ for prime digit count.}
    \label{fig:quadruplets}
\end{figure}

\subsection{Statistical Context}

Consider the probability of finding a quadruplet under a naive independence assumption.

At $n \approx 10^9$, we have $Q(n) \approx 47 \cdot n^{46} \approx 10^{415}$. The prime density is approximately:
\[
\rho \approx \frac{1}{\ln Q(n)} \approx \frac{1}{415 \cdot \ln 10} \approx \frac{1}{955}
\]

Under independence, the probability of four consecutive primes would be:
\[
P_{\text{naive}}(k=4) = \rho^4 \approx \left(\frac{1}{955}\right)^4 \approx 1.2 \times 10^{-12}
\]

Over $N = 2 \times 10^9$ trials, the expected count would be:
\[
E_{\text{naive}} = N \cdot P_{\text{naive}} \approx 2 \times 10^9 \times 1.2 \times 10^{-12} \approx 2.4 \times 10^{-3}
\]

The observed frequency is:
\[
\text{Observed} = \frac{14}{2 \times 10^9} \approx 7 \times 10^{-9}
\]

The ratio of observed to expected is:
\[
\frac{\text{Observed}}{E_{\text{naive}}/N} = \frac{7 \times 10^{-9}}{1.2 \times 10^{-12}} \approx 5{,}800
\]

\textbf{We observed 14 quadruplets, exceeding the naive expectation by a factor of approximately $10^4$.}

This notable excess is consistent with the Bateman-Horn correction factor $C_Q > 1$, which accounts for the structured dependencies introduced by the polynomial. Specifically, the correction factor $C_Q \approx 8.7$ for single primes compounds to approximately $C_Q^4 \approx 5{,}700$ for quadruplets, which closely matches the observed enhancement factor of $\approx 5{,}800$.

We emphasize that the naive independence model is used only as a baseline contrast; the Bateman--Horn correction provides the appropriate reference scale for comparison.

\subsection{Density Stability}

The consecutive pair density appears numerically stable across the search range:

\begin{table}[H]
\centering
\caption{Consecutive Pair Density by Range}
\label{tab:density}
\begin{tabular}{lrrr}
\toprule
Range & Pairs & Total Primes & Density \\
\midrule
$0 - 10^8$ & 5,533 & 518,223 & 1.068\% \\
$10^8 - 3 \times 10^8$ & 20,424 & 1,970,437 & 1.037\% \\
$3 \times 10^8 - 5 \times 10^8$ & 18,544 & 1,899,833 & 0.976\% \\
$5 \times 10^8 - 7 \times 10^8$ & 17,800 & 1,863,219 & 0.955\% \\
$7 \times 10^8 - 10^9$ & 25,869 & 2,746,257 & 0.942\% \\
$10^9 - 1.3 \times 10^9$ & 25,139 & 2,704,141 & 0.930\% \\
$1.3 \times 10^9 - 1.6 \times 10^9$ & 24,752 & 2,675,021 & 0.925\% \\
$1.6 \times 10^9 - 2 \times 10^9$ & 32,285 & 3,531,116 & 0.914\% \\
\midrule
\textbf{Total} & \textbf{170,346} & \textbf{17,908,247} & \textbf{0.951\%} \\
\bottomrule
\end{tabular}
\end{table}

The gradual decline from 1.07\% to 0.91\% is consistent with the Prime Number Theorem: as $Q(n)$ grows, prime density decreases as $1/\ln Q(n)$.

\begin{figure}[H]
    \centering
    \includegraphics[width=0.9\textwidth]{figure2_density.pdf}
    \caption{Consecutive pair density across the search range $n \le 2 \times 10^9$. The density shows a gradual decline from 1.07\% to 0.91\%, consistent with the Prime Number Theorem. The green dotted line indicates the mean density of 0.951\%.}
    \label{fig:density}
\end{figure}


% ============================================================
% 5. BATEMAN-HORN CONSISTENCY
% ============================================================
\section{Bateman-Horn Heuristic: Empirical Consistency}
\label{sec:bateman_horn}

\subsection{The Heuristic}

The Bateman-Horn conjecture \cite{bateman1962} predicts that for an irreducible polynomial $f(x)$ with positive leading coefficient:
\begin{equation}
\pi_f(N) = \#\{n \le N : f(n) \text{ is prime}\} \sim C_f \cdot \frac{N}{\ln f(N)}
\label{eq:bateman_horn}
\end{equation}
where the correction factor is:
\begin{equation}
C_f = \prod_{p \text{ prime}} \frac{1 - \omega_f(p)/p}{1 - 1/p}
\label{eq:correction_factor}
\end{equation}
and $\omega_f(p) = \#\{n \in \{0, 1, \ldots, p-1\} : f(n) \equiv 0 \pmod{p}\}$.

\subsection{Computing $C_Q$}

For $Q(n) = n^{47} - (n-1)^{47}$, we determine $\omega_Q(p)$ for small primes.

\begin{lemma}
\label{lem:omega_zero}
For all primes $p$ with $p \not\equiv 1 \pmod{47}$, we have $\omega_Q(p) = 0$.
\end{lemma}

\begin{proof}
Suppose $Q(n) \equiv 0 \pmod{p}$, i.e., $n^{47} \equiv (n-1)^{47} \pmod{p}$.

First, if $p \mid n$, then $Q(n) \equiv 0 - (-1)^{47} = 1 \pmod{p}$, so $Q(n) \not\equiv 0$. Similarly, if $p \mid (n-1)$, then $Q(n) \equiv 1 - 0 = 1 \pmod{p}$. Hence we may assume $\gcd(n(n-1), p) = 1$.

Let $r = n \cdot (n-1)^{-1} \pmod{p}$. Then $r^{47} \equiv 1 \pmod{p}$.

The order of $r$ divides both 47 and $p-1$. Since 47 is prime, $\ord_p(r) \in \{1, 47\}$.

If $\ord_p(r) = 1$, then $r \equiv 1$, implying $n \equiv n-1 \pmod{p}$, contradiction.

If $\ord_p(r) = 47$, then $47 \mid (p-1)$, i.e., $p \equiv 1 \pmod{47}$.

For primes $p \not\equiv 1 \pmod{47}$, no such $r$ exists.
\end{proof}

The smallest prime $p \equiv 1 \pmod{47}$ is $p = 283$.

\begin{lemma}
\label{lem:omega_46}
For all primes $p \equiv 1 \pmod{47}$, we have $\omega_Q(p) = 46$.
\end{lemma}

\begin{proof}
When $47 \mid (p-1)$, the group $\mathbb{F}_p^\times$ contains $47$ distinct $47$th roots of unity. The equation $z^{47} = 1$ where $z = n/(n-1)$ thus has 47 solutions in $\mathbb{F}_p$. The solution $z = 1$ corresponds to $n \to \infty$ (the projective point at infinity), leaving 46 finite solutions for $n$.
\end{proof}

Combining Lemmas \ref{lem:omega_zero} and \ref{lem:omega_46}:

\begin{equation}
C_Q = \prod_{p < 283} \frac{1}{1 - 1/p} \cdot \prod_{p \ge 283} \frac{1 - \omega_Q(p)/p}{1 - 1/p}
\end{equation}

\subsection{Numerical Estimate with Uncertainty}

We compute the truncated Euler product:
\begin{equation}
C(p_{\max}) = \prod_{p \le p_{\max}} \frac{1 - \omega_Q(p)/p}{(1 - 1/p)}
\end{equation}
which exhibits monotone stabilization as $p_{\max}$ increases. Based on the observed plateau, we adopt:
\begin{equation}
\boxed{C_Q = 8.7 \pm 0.1}
\label{eq:CQ_estimate}
\end{equation}
as a numerical estimate for subsequent comparisons.

\textbf{Important disclaimer:} No claim is made that this numerical convergence constitutes evidence for the validity of the Bateman-Horn conjecture itself. The agreement we observe is limited to the computational range $n \le 2 \times 10^9$.

\subsection{Source of the Large Correction Factor}

The value $C_Q \approx 8.7$ arises from the specific modular structure of $Q(n)$:

\begin{enumerate}
    \item \textbf{Modular lock at $\ell = 3$:} Since $Q(n) \equiv 1 \pmod{3}$ for all $n$, the polynomial never produces multiples of 3 (except $Q(1) = 1$). This removes the ``sieving by 3'' that affects random integers, contributing a factor of $\frac{3}{3-1} = 1.5$.
    
    \item \textbf{Absence of roots for small primes:} By Lemma \ref{lem:omega_zero}, $\omega_Q(p) = 0$ for all primes $p < 283$. Each such prime contributes a factor of $\frac{p}{p-1}$ to the product.
    
    \item \textbf{Shielding product:} The cumulative boost from all 60 shielding primes is:
    \[
    P_{\text{shield}} = \prod_{p < 283} \frac{p}{p-1} \approx 10.19
    \]
    
    \item \textbf{Splitting correction:} For primes $p \equiv 1 \pmod{47}$ (the first being $p = 283$, by Lemma \ref{lem:omega_46}), the polynomial has $\omega_Q(p) = 46$ roots, contributing factors $\frac{p-46}{p-1} < 1$. These corrections pull the product down from $\approx 10.19$ to the final value $C_Q \approx 8.70$.
\end{enumerate}

Intuitively, $Q(n)$ ``avoids'' small prime factors more effectively than random integers, leading to enhanced prime density. This is analogous to how Euler's polynomial $n^2 + n + 41$ produces many primes by avoiding small factors.

\subsection{Empirical Consistency}

We compare observed versus predicted prime counts:

\begin{table}[H]
\centering
\caption{Bateman-Horn Prediction vs.\ Observation}
\label{tab:bh_verification}
\begin{tabular}{lrr}
\toprule
& Observed & Predicted ($C_Q = 8.7 \pm 0.1$) \\
\midrule
$\pi_Q(2 \times 10^9)$ & 17,908,247 & $\sim 17{,}600{,}000 \pm 200{,}000$ \\
\midrule
Agreement & \multicolumn{2}{c}{Within $2\%$} \\
\bottomrule
\end{tabular}
\end{table}

\begin{figure}[H]
    \centering
    \includegraphics[width=0.85\textwidth]{figure3_ktuples.pdf}
    \caption{Distribution of polynomial prime $k$-tuples on a logarithmic scale. The steep decline from $k=2$ (170,346 pairs) to $k=4$ (14 quadruplets) is consistent with the compounding effect of prime density.}
    \label{fig:ktuples}
\end{figure}


% ============================================================
% 6. MINIMAL COUNTEREXAMPLE SEARCH
% ============================================================
\section{Minimal Counterexample Search}
\label{sec:counterexample}

To strengthen the empirical basis of our observations, we conducted a targeted search for deviations from the expected behavior.

\subsection{Methodology}

Define the empirical ratio:
\begin{equation}
R(N) = \frac{\pi_Q(N)}{C_Q \cdot N / \ln Q(N)}
\end{equation}

Rather than asserting convergence in a strict analytic sense, we observe that $R(N)$ appears numerically stable over the tested ranges, fluctuating within $\pm 5\%$ of unity.

\subsection{Search for Deviations}

For each $N$ up to $N_{\max} = 2 \times 10^9$, we computed:
\begin{equation}
\Delta R(N) = |R(N) - 1|
\end{equation}
and searched for intervals where $\Delta R(N) > \varepsilon$ for $\varepsilon = 0.15$.

\textbf{Result:} No such deviation was observed within the computational range. While this does not rule out larger-scale deviations, it suggests that any failure of the conjectured limiting behavior, if present, must occur beyond $n = 2 \times 10^9$.

This experiment should be regarded as a falsification attempt rather than confirmatory evidence, and its negative result serves only to delimit the scale at which deviations might plausibly arise.


% ============================================================
% 7. CONJECTURES
% ============================================================
\section{Conjectures}
\label{sec:conjectures}

Based on our empirical findings, we state the following conjectures in forms amenable to future theoretical or numerical investigation. \textbf{These conjectures are purely empirical and based on finite computational ranges.}

\begin{conjecture}[Polynomial Twin Prime Conjecture]
\label{conj:twin}
Let $R_2(N)$ denote the count of consecutive pairs $(n, n+1)$ with $n \le N$ such that both $Q(n)$ and $Q(n+1)$ are prime. Then:
\begin{equation}
\lim_{N \to \infty} R_2(N) = \infty
\end{equation}
\end{conjecture}

\textbf{Evidence:} 170,346 pairs observed with numerically stable density $\approx 0.95\%$ across all ranges.

\begin{conjecture}[Polynomial Quadruple Prime Conjecture]
\label{conj:quadruple}
There exist infinitely many positive integers $n$ such that $Q(n), Q(n+1), Q(n+2), Q(n+3)$ are all prime.
\end{conjecture}

\textbf{Evidence:} 14 quadruplets observed, with the largest at $n = 1{,}996{,}430{,}175$ near the search boundary, suggesting more exist beyond $n = 2 \times 10^9$.

\begin{conjecture}[Bateman-Horn Ratio Convergence]
\label{conj:ratio}
Let $R(N)$ denote the ratio of observed $Q(n)$ primes to the Bateman-Horn predicted count up to $N$. Then:
\begin{equation}
\lim_{N \to \infty} R(N) = 1
\end{equation}
\end{conjecture}

This conjecture is stated in a form amenable to future theoretical or numerical refutation. It should be interpreted as a statement about numerical stability rather than a claim of asymptotic truth.


% ============================================================
% 8. DISCUSSION AND LIMITATIONS
% ============================================================
\section{Discussion and Limitations}
\label{sec:conclusion}

\subsection{Summary of Findings}

Our analysis of $Q(n) = n^{47} - (n-1)^{47}$ over $n \le 2 \times 10^9$ reveals:

\begin{enumerate}
    \item \textbf{Exclusion:} Twin primes $(Q(n), Q(n)+2)$ are algebraically forbidden (Theorem \ref{thm:exclusion}).
    
    \item \textbf{Clustering:} Despite exclusion, prime quadruplets exist---14 instances of four consecutive generators all producing 400+ digit primes.
    
    \item \textbf{Consistency:} The correction factor $C_Q = 8.7 \pm 0.1$ yields predictions consistent with observations within $2\%$.
\end{enumerate}

\subsection{Limitations}

Several limitations should be noted:

\begin{itemize}
    \item All statistical observations are based on a finite computational range and may not persist as $N \to \infty$.
    
    \item The Bateman-Horn heuristic is not a theorem; our numerical agreement does not constitute a proof.
    
    \item Primality testing used probabilistic methods (Miller-Rabin with 25 rounds). While the probability of any reported prime being composite is negligibly small ($< 4^{-25}$), we cannot claim deterministic verification for primes of this magnitude ($> 10^{400}$).
\end{itemize}

\subsection{Implications for Bounded Gap Theory}

Zhang \cite{zhang2014} proved that $\liminf_{n \to \infty} (p_{n+1} - p_n) < 7 \times 10^7$, later improved to 246 by Maynard \cite{maynard2015} and Tao's Polymath project \cite{polymath2014}.

Our work complements these results by demonstrating:
\begin{itemize}
    \item Algebraic structure can \emph{forbid} specific gaps (the $+2$ exclusion).
    \item The same structure can permit clustering (quadruplets).
    \item Prime distribution in polynomial sequences exhibits ``channeling''---probability mass excluded from certain configurations may concentrate in others.
\end{itemize}

\subsection{Future Directions}

\begin{enumerate}
    \item \textbf{Extended search:} Computation to $n = 10^{10}$ may reveal quintuplets ($k=5$).
    
    \item \textbf{Generalization:} Study of $n^p - (n-1)^p$ for other primes $p$.
    
    \item \textbf{Theoretical framework:} Development of a modified GPY sieve incorporating polynomial constraints.
\end{enumerate}


% ============================================================
% 9. COMPREHENSIVE RESULTS OVERVIEW
% ============================================================
\section{Comprehensive Results Overview}

Figure \ref{fig:summary} presents a unified view of all major findings from this study.

\begin{figure}[H]
    \centering
    \includegraphics[width=0.98\textwidth]{mega_analysis_visualization.pdf}
    \caption{Comprehensive summary of the analysis of $Q(n) = n^{47} - (n-1)^{47}$ for $n \le 2 \times 10^9$. \textbf{Top-left:} Consecutive pair density showing stable $\approx 0.95\%$ across all ranges. \textbf{Top-right:} Key statistics including the 17.9 million primes, 170,346 pairs, 1,691 triples, and 14 quadruplets. \textbf{Bottom-left:} Location of the 14 quadruplets. \textbf{Bottom-right:} Summary of main conclusions.}
    \label{fig:summary}
\end{figure}


% ============================================================
% ACKNOWLEDGMENTS
% ============================================================
\section*{Acknowledgments}

The author thanks the developers of the GMP library and the open-source mathematical software community. Computational resources were provided by personal computing infrastructure.


% ============================================================
% DATA AVAILABILITY
% ============================================================
\section*{Data Availability and Reproducibility}

All numerical data used in this study, including raw constellation counts and interval statistics, are provided in supplementary CSV files. The complete dataset of 17,908,247 prime-generating $n$ values is available at:

\begin{center}
\url{https://github.com/Ruqing1963/Q47-Prime-Constellations}
\end{center}

Computations were performed using deterministic algorithms with fixed random seeds where applicable. The repository includes:
\begin{itemize}
    \item Complete list of $n$ values where $Q(n)$ is prime (17,908,247 entries)
    \item The 14 quadruplet starting values with verification data
    \item Analysis scripts and verification code
\end{itemize}


% ============================================================
% REFERENCES
% ============================================================
\bibliographystyle{plain}

\begin{thebibliography}{99}

\bibitem{bateman1962}
P.~T. Bateman and R.~A. Horn,
\newblock A heuristic asymptotic formula concerning the distribution of prime numbers,
\newblock \emph{Mathematics of Computation}, 16 (1962), 363--367.

\bibitem{green2008}
B.~Green and T.~Tao,
\newblock The primes contain arbitrarily long arithmetic progressions,
\newblock \emph{Annals of Mathematics}, 167 (2008), 481--547.

\bibitem{hardy1923}
G.~H. Hardy and J.~E. Littlewood,
\newblock Some problems of `Partitio Numerorum'; III: On the expression of a number as a sum of primes,
\newblock \emph{Acta Mathematica}, 44 (1923), 1--70.

\bibitem{maynard2015}
J.~Maynard,
\newblock Small gaps between primes,
\newblock \emph{Annals of Mathematics}, 181 (2015), 383--413.

\bibitem{polymath2014}
D.~H.~J. Polymath,
\newblock Variants of the Selberg sieve, and bounded intervals containing many primes,
\newblock \emph{Research in the Mathematical Sciences}, 1 (2014), Article 12.

\bibitem{zhang2014}
Y.~Zhang,
\newblock Bounded gaps between primes,
\newblock \emph{Annals of Mathematics}, 179 (2014), 1121--1174.

\bibitem{goldston2009}
D.~A. Goldston, J.~Pintz, and C.~Y. Y{\i}ld{\i}r{\i}m,
\newblock Primes in tuples I,
\newblock \emph{Annals of Mathematics}, 170 (2009), 819--862.

\bibitem{schinzel1958}
A.~Schinzel and W.~Sierpi\'{n}ski,
\newblock Sur certaines hypoth\`{e}ses concernant les nombres premiers,
\newblock \emph{Acta Arithmetica}, 4 (1958), 185--208.

\end{thebibliography}


% ============================================================
% APPENDIX
% ============================================================
\appendix

\section{Computational Details}
\label{app:computation}

\subsection{Hardware and Software}

\begin{itemize}
    \item \textbf{Search range:} $n \in [1, 2 \times 10^9]$
    \item \textbf{Hardware:} Computation performed on a workstation with AMD Ryzen 9 processor (16 cores) and 64GB RAM, with additional distributed processing across 4 nodes
    \item \textbf{Primality testing:} Fermat test (base 2) followed by Miller-Rabin with 25 rounds
    \item \textbf{Cross-validation:} All 14 quadruplets were cross-validated using a separate Python/gmpy2 implementation to ensure consistency with the primary C/GMP code
    \item \textbf{Software:} Primary implementation using GMP library for arbitrary-precision arithmetic; secondary validation using Python 3 with gmpy2
\end{itemize}

\subsection{Note on Primality Testing}

Given the magnitude of the primes involved ($Q(n) > 10^{400}$), deterministic primality proving via ECPP or AKS is computationally prohibitive. All candidates in this study passed 25 rounds of the Miller-Rabin probabilistic primality test, reducing the probability of any composite passing to negligible levels ($< 4^{-25} \approx 10^{-15}$).

For this reason, we use the term ``strong probable primes'' (also referred to informally as ``industrial-grade primes'' in the sense of passing extensive probabilistic verification) rather than ``proven primes.'' However, for integers of this form and magnitude, the probability that any of our reported primes are actually composite is astronomically small.

\section{Complete Quadruplet Data}

\begin{table}[H]
\centering
\caption{Extended Data for the 14 Quadruplets}
\begin{tabular}{rllll}
\toprule
\# & $n$ & $n+1$ & $n+2$ & $n+3$ \\
\midrule
1 & 117,309,848 & 117,309,849 & 117,309,850 & 117,309,851 \\
2 & 136,584,738 & 136,584,739 & 136,584,740 & 136,584,741 \\
3 & 218,787,064 & 218,787,065 & 218,787,066 & 218,787,067 \\
4 & 411,784,485 & 411,784,486 & 411,784,487 & 411,784,488 \\
5 & 423,600,750 & 423,600,751 & 423,600,752 & 423,600,753 \\
6 & 523,331,634 & 523,331,635 & 523,331,636 & 523,331,637 \\
7 & 640,399,031 & 640,399,032 & 640,399,033 & 640,399,034 \\
8 & 987,980,498 & 987,980,499 & 987,980,500 & 987,980,501 \\
9 & 1,163,461,515 & 1,163,461,516 & 1,163,461,517 & 1,163,461,518 \\
10 & 1,370,439,187 & 1,370,439,188 & 1,370,439,189 & 1,370,439,190 \\
11 & 1,643,105,964 & 1,643,105,965 & 1,643,105,966 & 1,643,105,967 \\
12 & 1,691,581,855 & 1,691,581,856 & 1,691,581,857 & 1,691,581,858 \\
13 & 1,975,860,550 & 1,975,860,551 & 1,975,860,552 & 1,975,860,553 \\
14 & 1,996,430,175 & 1,996,430,176 & 1,996,430,177 & 1,996,430,178 \\
\bottomrule
\end{tabular}
\end{table}

\end{document}
